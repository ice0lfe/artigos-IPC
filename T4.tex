% !TeX spellcheck = en_US
\documentclass[12pt]{article}

\usepackage{sbc-template}

\usepackage{graphicx,url}

\usepackage[brazil]{babel}   
%\usepackage[latin1]{inputenc}  
\usepackage[utf8]{inputenc}  
% UTF-8 encoding is recommended by ShareLaTex

     
\sloppy

\title{A Propriedade Intelectual}

\author{Edson Lemes da Silva\inst{1}, Lucas Cezar Parnoff\inst{1}  }

\address{Universidade Federal da Fronteira Sul
  (UFFS)\\
  Chapecó -- SC -- Brasil
\email{ed \textunderscore edsoon@hotmail.com, ice0life@gmail.com}
}

\begin{document} 

\maketitle

\begin{abstract}
  
\end{abstract}
     
\begin{resumo} 
  
\end{resumo}

\section{Introdução}\label{sec:introducao}

Este artigo foi desenvolvido, para avaliação
do curso de Iniciação da Pesquisa Científica, no qual
sua quarta avaliação é produzir um artigo sobre o assunto da
"Propriedade Intelectual", começando com as definições,
suas aplicações e entidades que estão contribuindo com
a propriedade intelectual.
\section{Definição}\label{sec:conteudo}

Este conceito pode ser entendido como a abrangência  dos direitos associados à sua invenção, ou seja, é a garantia de autoria sobre uma criação \cite{UFAL}, ou seja, é toda criação feita pelo autor, sendo que ele, quer a garantia de que seu trabalho, seja dele proprio, assim não permitindo algum uso, sem a previa permissão do criador.

Quando algo é criado, geralmente, o seu autor reivindica que os direitos sobre o feito sejam atribuídos ao seu nome. Justamente para evitar que alguém "roube" o trabalho idealizado. Além disso, também impede que o mesmo seja indevidamente utilizado, sem apresentar as devidas referências.

No momento em que uma invenção é registrada, ela passa a ser de uso exclusivo. O seu criador pode definir se, ela pode ou não ser utilizado por outras pessoas. Ele também pode definir quem pode ou não fazer uso. Deste modo, pode haver um retorno financeiro a partir disso.

Outro ponto importante, é que há uma restrição severa em relação a violação de uso de invenção, inclusive é tratado como crime perante a lei em vários países. Podemos citar como exemplo, a pirataria virtual \cite{PIRATE}, ou seja, downloads de arquivos de forma ilegal, tais como filmes, livros, músicas. Este é um problema grave, principalmente para os criadores destes arquivos, mesmo que apenas uma vez feito, já geram prejuízos financeiros. Já que é uma prática comum, é difícil manter o controle sobre isso.

\section{Aplicação da propriedade intelectual} \label{sec:aplicacao}

A partir desses conceitos, podemos ver como que são aplicados.

A principal delas é uma patente, que é utilizada para produtos do comércio,
assim concorrentes não podem copiar esse produto, ou parte dele, para conseguir retorno
financeiro, o unico problema da patente é que deve repassar todos os detalhes desse produto
para a entidade que faz o serviço de patente, colocando um risco de copiar partes desse
no desenvolvimento de produtos diferentes, apartir desses detalhes.

Outra muito aplicada é a marca, que a partir dela muitas empresas fazem a utilização de
produtos em varias linhas de produção para conseguir mais consumidores, porem varias marcas
são utilizadas somente em um tipo de produto, com o objetivo de lembrar do mesmo pela marca,
que normalmente é pequena e facil lembrar.


\section{Entidades Envolvidas}\label{sec:entidades}

Na região Sul em Santa Catarina a mais conhecida é a RENOVA marcas \& patentes,
segundo \cite{renova} trabalham junto com o instituto nacional da propriedade industrial(INPI)
principalmente nas patentes, no qual é feito o encaminhamento dos dados, para a criação 
da patente, que será utilizada pela depositante, que neste caso é uma pessoa juridica, 
mas pode ser uma pessoa fisica.

A CNPQ





 

\bibliographystyle{sbc}
\bibliography{sbc-tempkklate}
\end{document}
