% !TeX spellcheck = en_US
\documentclass[12pt]{article}

\usepackage{sbc-template}

\usepackage{graphicx,url}

\usepackage[brazil]{babel}   
%\usepackage[latin1]{inputenc}  
\usepackage[utf8]{inputenc}  
% UTF-8 encoding is recommended by ShareLaTex

     
\sloppy

\title{A Propriedade Intelectual}

\author{Edson Lemes da Silva\inst{1}, Lucas Cezar Parnoff\inst{1}  }

\address{Universidade Federal da Fronteira Sul
  (UFFS)\\
  Chapecó -- SC -- Brasil
\email{ed \textunderscore edsoon@hotmail.com, ice0life@gmail.com}
}

\begin{document} 

\maketitle

\begin{abstract}
  
\end{abstract}
     
\begin{resumo} 
  
\end{resumo}

\section{Introdução}\label{sec:introducao}

conteúdo de introdução

\section{Definição}\label{sec:conteudo}

Este conceito pode ser entendido como a abrangência  dos direitos associados à sua invenção, ou seja, é a garantia de autoria sobre uma criação \cite{UFAL}.

Quando algo é criado, geralmente, o seu autor reivindica que os direitos sobre o feito sejam atribuídos ao seu nome. Justamente para evitar que alguém "roube" o trabalho idealizado. Além disso, também impede que o mesmo seja indevidamente utilizado, sem apresentar as devidas referências.

No momento em que uma invenção é registrada, ela passa a ser de uso exclusivo. O seu criador pode definir se, ela pode ou não ser utilizado por outras pessoas. Ele também pode definir quem pode ou não fazer uso. Deste modo, pode haver um retorno financeiro a partir disso.

Outro ponto importante, é que há uma restrição severa em relação a violação de uso de invenção, inclusive é tratado como crime perante a lei em vários países. Podemos citar como exemplo, a pirataria virtual \cite{PIRATE}, ou seja, downloads de arquivos de forma ilegal, tais como filmes, livros, músicas. Este é um problema grave, principalmente para os criadores destes arquivos, uma fez que geram prejuízos. Já que é uma prática comum, é difícil manter o controle sobre isso.  
 


\subsection{Aplicação da propriedade intelectual} \label{sec:sub1}









 

\bibliographystyle{sbc}
\bibliography{sbc-tempkklate}
\end{document}
